\documentclass[notitlepage,abstract=on,twoside=semi]{scrartcl}

\usepackage{dandan}
\usepackage{amsmath,amsfonts,amssymb,amsthm}
\usepackage{mathtools}
\usepackage{comment}


\newcommand{\R}{\ensuremath{\mathbb{R}}}
\newcommand{\Z}{\ensuremath{\mathbb{Z}}}
\newcommand{\Q}{\ensuremath{\mathbb{Q}}}
\newcommand{\N}{\ensuremath{\mathbb{N}}}
\newcommand{\om}{\ensuremath{\Omega}}
\newcommand{\F}{\ensuremath{\mathcal{F}}}
\newcommand{\Prob}{\ensuremath{\mathbb{P}}}


\renewcommand*{\sectionformat}
{\color{purple}\S\thesection\autodot\enskip}

\begin{document}
\section{Lecture 1}
We begin with the definition of probability.
\begin{definition}[\textbf{Probability}]
  \label{1/5:1}
  A mathematical model about random experiments.
\end{definition}
In this class, we fix the notation $\om, \F, \Prob$. We say that $\om$ is our
``sample space'' or the set of all possible outcomes. For example, if our
experiment was flipping a coin, then $\om = \{H, T\}$. It is also important to
note that $\om$ can be countably or uncountably infinite. For example, our
experiment might be picking a positive integer in which $\om = \N$. Or our
experiment might be throwing a dart at a board in which $\om$ would be
uncountably infinite.

We define $\F$ to be the set of all events, mathematically speaking it's a
$\sigma$ algebra. We can define an event to be a subset of $\om$. For example,
if our experiment is flipping two coins, then an event could be that the first
coin is heads - the precise event would then be $\{HH, HT\}$ (notice that both
$HH$ and $HT$ are in $\om$.

Another interesting example might be where we roll a die as our
experiment. Then we could say an event is rolling an \textit{even} number
that's greater than 4. The precise event would be $\{4, 6\}$. Notice, however,
that we can actually write this event as the intersection of two events. That
is
\begin{enumerate}
\item Rolling an even number : $\{2,4,6\}$.
\item Rolling a number greater than 4 : $\{4,5,6\}$.
\end{enumerate}
Then the intersection of these two events is exactly our event of rolling an
even number greater than 4.
\begin{remark}
  When we say one event $A$ happens, it means we get some outcome $\omega \in
  \om$ s.t. $\omega \in A$. If $\omega \not \in A$, then $A$ \textit{doesn't}
  happen.
\end{remark}
\begin{remark}
  We can also do set operations on events. That is if $A$ and $B$ are events,
  then we can say that the event of $A$ or $B$ happening is exactly $A \cup B$.
\end{remark}
\begin{definition}
  We define the complement of event $A$ to be $A^{c} = \{\omega \in \om :
  \omega \not \in A\}$.
\end{definition}
\begin{remark}
  If $\om$ is finite, then one possible choice of $\F$ is $\F = 2^{\om}$.
\end{remark}
Note that $2^{\om}$ is defined to be the \textit{power set} of $\om$, or the
``set of all subsets of $\om$''.

One more thing to note, but not terribly important for this class is that if
$\om$ is infinite, then $\F \neq 2^{\om}$.

Finally, we define $\Prob$ to be a function from $\F$ to the closed interval
$[0, 1]$. That is, $\Prob : \F \to [0, 1]$. Notice that this means we can only
talk about the probability of one event.

If we want to talk about the probability of some outcome $\omega$, then
formally we would define the event $\{\omega\}$ and ask what is
$\Prob(\{\omega\})$.

Recall that we said that $\F$ is a ``$\sigma$ algebra''. Here we define it
formally.
\begin{definition}
  Let $\om$ be a set. We say that $\F$ is a $\sigma$ algebra of $\om$ if it
  satisfies:
  \begin{enumerate}
  \item $\emptyset \in \F$
  \item If $A_{1}, A_{2}, \cdots \in \F$, then $\bigcup_{i=1}^{\infty} A_{i}
    \in \F$
  \item If $A \in \F$, then $A^{c} \in F$ (recall that $A^{c}$ is the
    \textit{complement} if $A$)
  \end{enumerate}
\end{definition}
Now just from these three properties, we can, in fact, prove that if $A_{1},
A_{2}, \cdots, A_{n} \in \F$, then $\bigcap_{i=1}^{n} A_{i} \in \F$ as well.
\end{document}





